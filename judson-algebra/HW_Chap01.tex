\documentclass[12pt]{amsart}

%Below are some necessary packages for your course.
\usepackage{amsfonts,latexsym,amsthm,amssymb,amsmath,amscd,euscript}
\usepackage{framed}
\usepackage{fullpage}
\usepackage{hyperref}
\hypersetup{colorlinks=true,citecolor=blue,urlcolor =black,linkbordercolor={1 0 0}}
\usepackage{enumitem}
    
\newenvironment{statement}[1]{\smallskip\noindent\color[rgb]{1.00,0.00,0.50} {\bf #1.}}{}
\allowdisplaybreaks[1]

%Below are the theorem, definition, example, lemma, etc. body types.

\newtheorem{theorem}{Theorem}
\newtheorem*{proposition}{Proposition}
\newtheorem{lemma}[theorem]{Lemma}
\newtheorem{corollary}[theorem]{Corollary}
\newtheorem{conjecture}[theorem]{Conjecture}
\newtheorem{postulate}[theorem]{Postulate}
\theoremstyle{definition}
\newtheorem{defn}[theorem]{Definition}
\newtheorem{example}[theorem]{Example}

\theoremstyle{remark}
\newtheorem*{remark}{Remark}
\newtheorem*{notation}{Notation}
\newtheorem*{note}{Note}

% You can define new commands to make your life easier.
\newcommand{\BR}{\mathbb R}
\newcommand{\BC}{\mathbb C}
\newcommand{\BF}{\mathbb F}
\newcommand{\BQ}{\mathbb Q}
\newcommand{\BZ}{\mathbb Z}
\newcommand{\BN}{\mathbb N}

% We can even define a new command for \newcommand!
\newcommand{\nc}{\newcommand}

% If you want a new function, use operatorname to define that function (don't use \text)
\nc{\on}{\operatorname}
\nc{\Spec}{\on{Spec}}

\title{Judson Chapter $1$ Exercises}
\date{\today}

\begin{document}

\maketitle

\vspace*{-0.25in}
\centerline{Hank Alme}
\centerline{\href{mailto:almehj@alumni.rice.edu}{{\tt almehj@alumni.rice.edu}}}
\vspace*{0.15in}


\begin{statement}{1}
  Suppose that
  \begin{align*}
    A &= \{ x : x \in \BN\  \text{and $x$ is even} \} \\
    B &= \{ x : x \in \BN\  \text{and $x$ is prime} \} \\
    C &= \{ x : x \in \BN\  \text{and $x$ is a multiple of $5$} \}
  \end{align*}
  Describe each of the following sets.
  
\end{statement}
\begin{enumerate}[label=(\alph*)]
\item $A \cap B = \{2\}$
\item $B \cap C = \{5\}$
\item $A \cup B = \{ x : \text{$x$ is even or $x$ is prime} \}$
\item $A \cap (B \cup C) = \{2\}$
\end{enumerate}


\begin{statement}{2}
  If $A = \{a,b,c\}$, $B = \{1,2,3\}$,$C= {x}$, and $D = \emptyset$,
  list all of the elements in each of the following sets.
\end{statement}

\begin{enumerate}[label=(\alph*)]
  \item $A \times B =
    \{(a,1),(a,2),(a,3),(b,1),(b,2),(b,3),(c,1),(c,2),(c,3)\} $
  \item $B \times A = \{
    (1,a),(1,b),(1,c),(2,a),(2,b),(2,c),(3,a),(3,b),(3,c) \}$
  \item $A \times B \times C = \{(a,1,x),(a,2,x),(a,3,x),(b,1,x),(b,2,x),(b,3,x), \\
    (c,1,x),(c,2,x),(c,3,x)\} $
      \item $A \times D = \emptyset$
\end{enumerate}


\begin{statement}{3}
  Find an example of two nonempty sets $A$ and $B$ for which $A \times B = B \times A$ is true.
\end{statement}

$A = B$ is the only way for nonempty sets to have symmetric Cartesian
products.
\begin{proof}
Let $A$ and $B$ be two such nonempty sets. By the definition of Cartesian product:
\begin{align*}
  A \times B &= \{ (a,b) : a \in A \text{ and } b \in B \} \\
  B \times A &= \{ (b,a) : a \in A \text{ and } b \in B \}
\end{align*}
Since $A$ and $B$ are nonempty, the two Cartesian products in question
will also be nonempty. Let $(a,b) \in A \times B$. Since $A \times B =
B \times A$, $(a,b) \in B \times A$. By the definition of $B \times A$
above, that means that $a \in B$ and $b \in A$. The definition of $A
\times B$ thus means that for all $a \in A, a \in B$ also, thus $A
\subset B$. Similarly, we see that for all $b \in B, b \in A$ so $B
\subset A$. Therefore $A=B$.
\end{proof}


\begin{statement}{4}
Prove $A \cup \emptyset = A$ and $A \cap \emptyset = \emptyset $
\end{statement}
\begin{proof}
  For the first proposition, observe that there is no $x$ such that $x \in \emptyset$, so
  \begin{align*}
    A \cup \emptyset &= \{ x : x \in A \text{ or } x \in \emptyset \} \\
    &= \{ x : x \in A \} \\
    &= A
  \end{align*}

  For the second, observe
  \begin{align*}
    A \cap \emptyset &= \{ x : x \in A \text{ and } x \in \emptyset \} \\
    &= \emptyset
  \end{align*}
  since for any $x \in A$, it cannot be that $x \in \emptyset$, making
  the condition impossible to fulfill.
\end{proof}

\begin{proof}
For $A \cup \emptyset$: let $a \in A \cup \emptyset$. By definition of
union $a \in A$ or $a \in \emptyset$. Since $\emptyset$ has no
members, $a \notin \emptyset$, therefore $a \in A$, therefore $A \cup
\emptyset \subset A$. Now let $a' \in A$, by definition of union, $a'
\in A \cup \emptyset$, so $A \subset A \cup \emptyset$, proving the
first proposition.

For $A \cap \emptyset$: if $A = \emptyset$, neither set has any
members thus the proposition is trivially true since there are no
possible members of $A \cap \emptyset$, so assume $A$ is nonempty and
let $a \in A$. By definition of intersection $a \in A \cap \emptyset$
requires $a \in A$ (true) and $a \in \emptyset$, which is impossible
since $\emptyset$ has no members, thus no member of $A$ is in $A \cap
\emptyset$. Since $\emptyset$ has no members, here is no possible
member for the intersection, so $A \cap \emptyset = \emptyset$.
\end{proof}


\begin{statement}{5}
Prove $A \cup B = B \cup A$ and $A \cap B = B \cap A$.
\end{statement}
\begin{proof}
Using the definitions of union and intersection
\begin{align*}
  A \cup B &= \{x : x \in A \text{ or } x \in B\} \\
    &= \{x : x \in B \text{ or } x \in A\} \\
      &= B \cup A
\end{align*}
and
\begin{align*}
  A \cap B &= \{x : x \in A \text{ and } x \in B\} \\
    &= \{x : x \in B \text{ and } x \in A\} \\
      &= B \cap A
\end{align*}

\end{proof}

\begin{proof}
Union: let $ x \in A \cup B$, then by definition of union $x \in A$ or
$x \in B$, which means $x \in B \cup A$ also by definition of union, thus $A
\cup B \subset B \cup A$. Similary it can be shown that $B \cup A
\subset A \cup B$, proving the proposition.

Intersection is handled similarly: let $x \in A \cap B$, then $x \in
A$ and $x \in B$, meeting the definition for $B \cap A$, so $A \cap B
\subset B \cap A$. Considering $y \in B \cap A$ shows that $B \cap A
\subset A \cap B$, completing the proof.
\end{proof}


\begin{statement}{6}
Prove $A \cup (B \cap C) = (A \cup B) \cap (A \cup C)$.
\end{statement}

\begin{proof}
  If $A = \emptyset$, we can use the results from Problem 4 to get $A
  \cup (B \cap C) = B \cap C$ and $(A \cup B) \cap (A \cup C) = B \cap
  C$, proving the proposition.

  We can also use Problem 4 if $B = \emptyset$. Considering the left
  hand side of the proposition: $(B \cap C) = \emptyset$ so $A \cup (B
  \cap C) = A$. Considering the right hand side of the proposition,
  $(A \cup B) \cap (A \cup C) = A \cap (A \cup C)$. For $a \in A$, $a
  \in (A \cup C)$ so $A \subset A \cap (A \cup C)$, and for $a` in A
  \cap (A \cup C)$, $a \in A$ by definition of intersection, so $A
  \cap (A \cup C) \subset A$, giving $A \cap (A \cup C) = A$ and
  proving the proposition. Similary, it can be shown that the
  proposition is true if $C = \emptyset$.

  So assume $A, B,$ and $C$ are all nonempty.
  
  Let $x \in A \cup (B \cap C)$, then $x \in A$ or $x \in (B \cup
  C)$. If $x \in A$, then $x \in (A \cup B)$ and $x \in (A \cup C)$,
  thus $x \in (A \cup B) \cap (A \cup C)$. If $x \in (B \cup C)$, then
  $x \in B$ and $x in C$, thus $x \in (A cup B)$ and $x \in (A \cup
  C)$, so again $x \in (A \cup B) \cap (A \cup C)$. Therefore $A \cup
  (B \cap C) \subset (A \cup B) \cap (A \cup C)$.
\end{proof}


\begin{statement}{7}
Prove $A \cap (B \cup C) = (A \cup B) \cap (A \cup C)$.
\end{statement}




\begin{statement}{8}
Prove $A \subset B$ if and only if $A \cap B = A$
\end{statement}
To be worked out


\begin{statement}{9}
Prove $(A \cap B)' = A' \cup B'$.
\end{statement}
To be worked out


\begin{statement}{10}
Prove $A \cup B = (A \cap B) \cup (A \setminus B) \cup (B \setminus A)$.
\end{statement}
To be worked out


\begin{statement}{11}
Prove $(A \cup B) \times C = (A \times C) \cup (B \times C)$.
\end{statement}
To be worked out


\begin{statement}{12}
Prove $(A \cap B) \setminus B = \emptyset$.
\end{statement}
To be worked out


\begin{statement}{13}
Prove $(A \cup B) \setminus B = A \setminus B$.
\end{statement}
To be worked out


\begin{statement}{14}
Prove $A \setminus (B \cup C) = (A \setminus B) \cap (A \setminus C)$.
\end{statement}
To be worked out


\begin{statement}{15}
Prove $A \cap (B \setminus C) = (A \cap B) \setminus (A \cap C)$.
\end{statement}
To be worked out


\begin{statement}{16}
Prove $(A \setminus B) \cup (B \setminus A) = (A \cup B) \ (A \cap B)$.
\end{statement}
To be worked out


\begin{statement}{17}
Which of the following relations $f : \BQ \to \BQ$ define a mapping? In each case, supply a reason why $f$ is or is not a mapping.
\end{statement}

\begin{flalign*}
  &\text{(a) }f(p/q) = \frac{p+1}{p -2} &
\end{flalign*}

\begin{flalign*}
  &\text{(b) }f(p/q) = \frac{3p}{3q} &
\end{flalign*}

\begin{flalign*}
  &\text{(c) }f(p/q) = \frac{p+q}{q^2} &
\end{flalign*}

\begin{flalign*}
  &\text{(d) }f(p/q) = \frac{3p^2}{7q^2} - \frac{p}{q}&
\end{flalign*}


\begin{statement}{18}
Determine which of the following functions are one-to-one and which
are onto. If the function is not onto, determine its range.
\end{statement}

\begin{enumerate}[label=(\alph*)]
\item $f : \BR \to \BR$ defined by $f(x) = e^x$
\item $f : \BZ \to \BZ$ defined by $f(n) = n^2 + 3$
\item $f : \BR \to \BR$ defined by $f(x) = \sin x$
\item $f : \BZ \to \BZ$ defined by $f(x) = x^2$
\end{enumerate}


\begin{statement}{19}
Let $f : A \to B$ and $g : B \to C$ be invertible mappings; that is,
mappings such that $f^{-1}$ and $g^{-1}$ exist. Show that $(g \circ
f)^{-1} = f^{-1} \circ g^{-1}$.
\end{statement}
To be worked out 


\begin{statement}{20}
Define the following functions:
\end{statement}

\begin{enumerate}[label=(\alph*)]
\item $f : \BN \to \BN$ that is one-to-one but not onto.
\item $f : \BN \to \BN$ that is onto but not one-to-one.
\end{enumerate}

\begin{statement}{21}
Prove that the relation defined on $\BR^2$ by $(x_1,y_1) \sim (x_2,y_2)$ if $x_1^2 + y_1^2 = x_2^2 + y_2^2$ is an equivalence relation.
\end{statement}
To be worked out


\begin{statement}{22}
To be entered
\end{statement}
To be worked out


\begin{statement}{23}
To be entered
\end{statement}
To be worked out


\begin{statement}{24}
To be entered
\end{statement}
To be worked out


\begin{statement}{25}
To be entered
\end{statement}
To be worked out


\begin{statement}{26}
To be entered
\end{statement}
To be worked out


\begin{statement}{27}
To be entered
\end{statement}
To be worked out


\begin{statement}{28}
To be entered
\end{statement}
To be worked out


\begin{statement}{29}
Projective Real line: define a relationship on $\BR^2 \setminus (0,0)$
by letting $(x_1,y_1) \sim (x_2,y_2)$ if there exists a nonzero real
number $\lambda$ such that $(x_1,y_1) \sim (\lambda x_2,\lambda
y_2)$. Prove that $\sim$ defines an equivalence relation on
$\BR^2 \setminus (0,0)$. What are the corresponding equivalence
classes? This equivalence relation defines the projective line,
denoted $\mathbb{P}(\BR)$, which is very important in geometry.

\end{statement}

foo

bar

baz



\end{document}
