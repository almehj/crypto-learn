\documentclass[12pt]{amsart}

%Below are some necessary packages for your course.
\usepackage{amsfonts,latexsym,amsthm,amssymb,amsmath,amscd,euscript}
\usepackage{framed}
\usepackage{fullpage}
\usepackage{hyperref}
\hypersetup{colorlinks=true,citecolor=blue,urlcolor =black,linkbordercolor={1 0 0}}
\usepackage{enumitem}
    
\newenvironment{statement}[1]{\smallskip\noindent\color[rgb]{1.00,0.00,0.50} {\bf #1.}}{}
\allowdisplaybreaks[1]

%Below are the theorem, definition, example, lemma, etc. body types.

\newtheorem{theorem}{Theorem}
\newtheorem*{proposition}{Proposition}
\newtheorem{lemma}[theorem]{Lemma}
\newtheorem{corollary}[theorem]{Corollary}
\newtheorem{conjecture}[theorem]{Conjecture}
\newtheorem{postulate}[theorem]{Postulate}
\theoremstyle{definition}
\newtheorem{defn}[theorem]{Definition}
\newtheorem{example}[theorem]{Example}

\theoremstyle{remark}
\newtheorem*{remark}{Remark}
\newtheorem*{notation}{Notation}
\newtheorem*{note}{Note}

% You can define new commands to make your life easier.
\newcommand{\BR}{\mathbb R}
\newcommand{\BC}{\mathbb C}
\newcommand{\BF}{\mathbb F}
\newcommand{\BQ}{\mathbb Q}
\newcommand{\BZ}{\mathbb Z}
\newcommand{\BN}{\mathbb N}

% We can even define a new command for \newcommand!
\newcommand{\nc}{\newcommand}

% If you want a new function, use operatorname to define that function (don't use \text)
\nc{\on}{\operatorname}
\nc{\Spec}{\on{Spec}}

\title{Judson Chapter $2$ Exercises}
\date{\today}

\begin{document}

\maketitle

\vspace*{-0.25in}
\centerline{Hank Alme}
\centerline{\href{mailto:almehj@alumni.rice.edu}{{\tt almehj@alumni.rice.edu}}}
\vspace*{0.15in}


\begin{statement}{1}
  Prove that
  \begin{equation*}
    1^2 + 2^2 + \cdots + n^2 = \frac{n(n+1)(2n+1)}{6}
  \end{equation*}
  for $n \in \BN$.
\end{statement}
\begin{proof}
  For $n =1$ we have $1^2 = 1$ and
  \begin{align*}
    \frac{n(n+1)(2n+1)}{6} &= \frac{1(1+1)(2+1)}{6} \\
    &= \frac{1 \cdot 2 \cdot 3}{6} \\
    &= \frac{6}{6} \\
    &= 1
  \end{align*}
  Assume the proposition holds for $n$. Then
  \begin{align*}
    1^2 + 2^2 + \cdots + n^2 + (n+1)^2 &= \frac{n(n+1)(2n+1)}{6} + (n+1)^2 \\
    &= \frac{n(n+1)(2n+1) + 6(n+1)^2}{6}  \\
    &= \frac{(n+1)[n(2n+1) + 6(n+1)]}{6}  \\
    &= \frac{(n+1)[2n^2+n + 6n+6]}{6}  \\
    &= \frac{(n+1)[2n^2+7n +6]}{6}  \\
    &= \frac{(n+1)(n+2)(2n +3)}{6}  \\
    &= \frac{(n+1)(n+2)(2(n+1) + 1)}{6}  
  \end{align*}
  Showing that the proposition also holds for $n+1$.
\end{proof}

\begin{statement}{2}
  Prove that
  \begin{equation*}
        1^3 + 2^3 + \cdots + n^3 = \frac{n^2(n+1)^2}{4}
  \end{equation*}
  for $n \in \BN$.
\end{statement}
\begin{proof}
  For $n=1$ we have $n^3 = 1$ and 
  \begin{align*}
    \frac{n^2(n+1)^2}{4} &= \frac{1^2(2)^2}{4} \\
    &= \frac{1 \cdot 4}{4} \\
    &= 1
  \end{align*}
  So assume the proposition holds for $n$. For $n+1$
  \begin{align*}
    1^3 + 2^3 + \cdots + n^3 + (n+1)^3 &= \frac{n^2(n+1)^2}{4} + (n+1)^3 \\
    &= \frac{n^2(n+1)^2 + 4(n+1)^3}{4} \\
    &= \frac{(n+1)^2 [n^2 + 4(n+1)]}{4} \\
    &= \frac{(n+1)^2 (n^2 + 4n+4)}{4} \\
    &= \frac{(n+1)^2 (n + 2)^2}{4}
  \end{align*}
  Showing the proposition holds for $n+1$.
\end{proof}

\begin{statement}{3}
  Prove that $n! > 2^n$ for $n \ge 4$.
\end{statement}
\begin{proof}
  For $n=4$, $n! = 4 \cdot 3\cdot 2\cdot 1 = 24$ and $2^4 = 16 < 24$,
  establishing the base case.

  Assume the proposition holds for $n$, so 
  \begin{align*}
    n! &> 2^n \\
    (n+1) n! &> (n+1) 2^n \\
    (n+1)! &> (n+1) 2^n \\
    &> 2^{n+1}
  \end{align*}
  since $n > 2$. 
\end{proof}


\begin{statement}{4}
  Prove that
  \begin{align*}
    x + 4x + 7x + \cdots + (3n-2)x = \frac{n(3n-1)x}{2}
  \end{align*}
  for $n \in \BN$.
\end{statement}
\begin{proof}
  Consider $n=1$, the left hand side will be $x$, and
  \begin{align*}
    \frac{n(3n-1)x}{2} &= \frac{1(1\cdot 3 -1)x}{2} \\
    &= \frac{1(2)x}{2} \\
    &= x
  \end{align*}
  establishing the base case. Now assume the proposition true for n.

  \begin{align*}
    \left (\sum_{i=1}^n (3i-2)x \right ) + (3(n+1) -2)x &= \frac{n(3n-1)x}{2}+(3(n+1) -2)x\\
    &= \frac{n(3n-1)x +2(3(n+1) -2)x}{2}\\
    &= \frac{[n(3n-1) +(6(n+1) -4)]x}{2}\\
    &= \frac{(3n^2+5n+2)x}{2}\\
    &= \frac{(n+1)(3n+2)x}{2}\\
    &= \frac{(n+1)(3(n+1) - 1)x}{2}
  \end{align*}
  Showing the proposition holds for $n+1$.
\end{proof}

\begin{statement}{5}
  Prove that $10^{n+1} + 10^n + 1$ is divisible by $3$ for $n \in \BN$.
\end{statement}
\begin{proof}
  Consider $n=1$, then we have $10^2 + 10 + 1 = 111 = 3\cdot 37$, establishing the base case.

  Now assume $f(n) = 10^{n+1} + 10^n + 1$ is divisible by $3$, we need
  to show $f(n+1) = 10^{n+2} + 10^{n+1} + 1$ is divisible by
  $3$. Since $3 \mid f(n)$, we know there is an $m \in \BZ$ such that
  $f(n) = 3m$. Thus $f(n) = 10^{n+1} + 10^n + 1 = 3m$. Multiplying
  both sides by $10$ gives
  \begin{align*}
    10^{n+2} + 10^{n+1} + 10 &= 30m \\
    10^{n+2} + 10^{n+1} + 1 + 9 &= 30m \\
    10^{n+2} + 10^{n+1} + 1  &= 30m - 9 \\
    f(n+1)  &= 3(10m - 3)
  \end{align*}
  meaning $3 \mid f(n+1)$, proving the proposition.
\end{proof}

\begin{statement}{6}
Prove that $f(n) = 4\cdot 10^{2n} + 9\cdot 10^{2n-1} + 5$ is divisible
by $99$ for $n \in \BN$.
\end{statement}
\begin{proof}
  Consider $n=1$, then $f(1) = 4\cdot 10^2 + 9\cdot 10^1 + 5 = 495 =
  99\cdot 5$, establishing the base case.

  Now assume $99 \mid f(n)$, so there is an $m \in \BZ$ such that $f(n) = 99m$. Similar to problem 5:
  \begin{align*}
    4\cdot 10^{2n} + 9\cdot 10^{2n-1} + 5 &= 99m \\
    100(4\cdot 10^{2n} + 9\cdot 10^{2n-1} + 5) &= 99\cdot 100m \\
    4\cdot 10^{2n + 2} + 9\cdot 10^{2n +1} + 500 &= 99\cdot 100m \\
    4\cdot 10^{2(n + 1)} + 9\cdot 10^{2(n+1) - 1} + 5+ 495 &= 99\cdot 100m \\
    f(n+1) + 5\cdot99 &= 99\cdot 100m \\
    f(n+1) &= 99\cdot(100m -5)
  \end{align*}
  meaning $99 \mid f(n+1)$, proving the proposition.
\end{proof}

\begin{statement}{7}
  Show that
  \begin{equation*}
    \sqrt[n]{a_1 a_2 \cdots a_n} \le \frac{1}{n} \sum_{k=1}^n a_k
    \end{equation*}
\end{statement}

\begin{statement}{8}
  Prove the Leibnitz rule for $f^{(n)}(x)$, where $f^{(n)}$ is the $n$th derivative of $f$; that is, show that
  \begin{equation*}
    (fg)^{(n)}(x) = \sum_{k=0}^n \binom{n}{k} f^{(k)}(x) g^{(n-k)}(x).
    \end{equation*}
\end{statement}
\begin{proof}
  if $n=0$, then the proposition reduces to $(fg)(x) = f(x)g(x)$,
  establishing the base case. Assume the rule is true for $n$.

  
\end{proof}
\end{document}
