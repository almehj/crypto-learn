\documentclass[12pt]{amsart}

%Below are some necessary packages for your course.
\usepackage{amsfonts,latexsym,amsthm,amssymb,amsmath,amscd,euscript}
\usepackage{framed}
\usepackage{fullpage}
\usepackage{hyperref}
\hypersetup{colorlinks=true,citecolor=blue,urlcolor =black,linkbordercolor={1 0 0}}
\usepackage{enumitem}
    
\newenvironment{statement}[1]{\smallskip\noindent\color[rgb]{1.00,0.00,0.50} {\bf #1.}}{}
\allowdisplaybreaks[1]

%Below are the theorem, definition, example, lemma, etc. body types.

\newtheorem{theorem}{Theorem}
\newtheorem*{proposition}{Proposition}
\newtheorem{lemma}[theorem]{Lemma}
\newtheorem{corollary}[theorem]{Corollary}
\newtheorem{conjecture}[theorem]{Conjecture}
\newtheorem{postulate}[theorem]{Postulate}
\theoremstyle{definition}
\newtheorem{defn}[theorem]{Definition}
\newtheorem{example}[theorem]{Example}

\theoremstyle{remark}
\newtheorem*{remark}{Remark}
\newtheorem*{notation}{Notation}
\newtheorem*{note}{Note}

% You can define new commands to make your life easier.
\newcommand{\BR}{\mathbb R}
\newcommand{\BC}{\mathbb C}
\newcommand{\BF}{\mathbb F}
\newcommand{\BQ}{\mathbb Q}
\newcommand{\BZ}{\mathbb Z}
\newcommand{\BN}{\mathbb N}

% We can even define a new command for \newcommand!
\newcommand{\nc}{\newcommand}

% If you want a new function, use operatorname to define that function (don't use \text)
\nc{\on}{\operatorname}
\nc{\Spec}{\on{Spec}}

\title{Judson Chapter $2$ Exercises}
\date{\today}

\begin{document}

\maketitle

\vspace*{-0.25in}
\centerline{Hank Alme}
\centerline{\href{mailto:almehj@alumni.rice.edu}{{\tt almehj@alumni.rice.edu}}}
\vspace*{0.15in}


\begin{statement}{1}
  Prove that
  \begin{equation*}
    1^2 + 2^2 + \cdots + n^2 = \frac{n(n+1)(2n+1)}{6}
  \end{equation*}
  for $n \in \BN$.
\end{statement}
\begin{proof}
  For $n =1$ we have $1^2 = 1$ and
  \begin{align*}
    \frac{n(n+1)(2n+1)}{6} &= \frac{1(1+1)(2+1)}{6} \\
    &= \frac{1 \cdot 2 \cdot 3}{6} \\
    &= \frac{6}{6} \\
    &= 1
  \end{align*}
  Assume the proposition holds for $n$. Then
  \begin{align*}
    1^2 + 2^2 + \cdots + n^2 + (n+1)^2 &= \frac{n(n+1)(2n+1)}{6} + (n+1)^2 \\
    &= \frac{n(n+1)(2n+1) + 6(n+1)^2}{6}  \\
    &= \frac{(n+1)[n(2n+1) + 6(n+1)]}{6}  \\
    &= \frac{(n+1)[2n^2+n + 6n+6]}{6}  \\
    &= \frac{(n+1)[2n^2+7n +6]}{6}  \\
    &= \frac{(n+1)(n+2)(2n +3)}{6}  \\
    &= \frac{(n+1)(n+2)(2(n+1) + 1)}{6}  
  \end{align*}
  Showing that the proposition also holds for $n+1$.
\end{proof}

\begin{statement}{2}
  Prove that
  \begin{equation*}
        1^3 + 2^3 + \cdots + n^3 = \frac{n^2(n+1)^2}{4}
  \end{equation*}
  for $n \in \BN$.
\end{statement}
\begin{proof}
  For $n=1$ we have $n^3 = 1$ and 
  \begin{align*}
    \frac{n^2(n+1)^2}{4} &= \frac{1^2(2)^2}{4} \\
    &= \frac{1 \cdot 4}{4} \\
    &= 1
  \end{align*}
  So assume the proposition holds for $n$. For $n+1$
  \begin{align*}
    1^3 + 2^3 + \cdots + n^3 + (n+1)^3 &= \frac{n^2(n+1)^2}{4} + (n+1)^3 \\
    &= \frac{n^2(n+1)^2 + 4(n+1)^3}{4} \\
    &= \frac{(n+1)^2 [n^2 + 4(n+1)]}{4} \\
    &= \frac{(n+1)^2 (n^2 + 4n+4)}{4} \\
    &= \frac{(n+1)^2 (n + 2)^2}{4}
  \end{align*}
  Showing the proposition holds for $n+1$.
\end{proof}

\begin{statement}{3}
  Prove that $n! > 2^n$ for $n \ge 4$.
\end{statement}
\begin{proof}
  For $n=4$, $n! = 4 \cdot 3\cdot 2\cdot 1 = 24$ and $2^4 = 16 < 24$,
  establishing the base case.

  Assume the proposition holds for $n$, so 
  \begin{align*}
    n! &> 2^n \\
    (n+1) n! &> (n+1) 2^n \\
    (n+1)! &> (n+1) 2^n \\
    &> 2^{n+1}
  \end{align*}
  since $n > 2$. 
\end{proof}


\begin{statement}{4}
  Prove that
  \begin{align*}
    x + 4x + 7x + \cdots + (3n-2)x = \frac{n(3n-1)x}{2}
  \end{align*}
  for $n \in \BN$.
\end{statement}
\begin{proof}
  Consider $n=1$, the left hand side will be $x$, and
  \begin{align*}
    \frac{n(3n-1)x}{2} &= \frac{1(1\cdot 3 -1)x}{2} \\
    &= \frac{1(2)x}{2} \\
    &= x
  \end{align*}
  establishing the base case. Now assume the proposition true for n.

  \begin{align*}
    \left (\sum_{i=1}^n (3i-2)x \right ) + (3(n+1) -2)x &= \frac{n(3n-1)x}{2}+(3(n+1) -2)x\\
    &= \frac{n(3n-1)x +2(3(n+1) -2)x}{2}\\
    &= \frac{[n(3n-1) +(6(n+1) -4)]x}{2}\\
    &= \frac{(3n^2+5n+2)x}{2}\\
    &= \frac{(n+1)(3n+2)x}{2}\\
    &= \frac{(n+1)(3(n+1) - 1)x}{2}
  \end{align*}
  Showing the proposition holds for $n+1$.
\end{proof}


\end{document}
